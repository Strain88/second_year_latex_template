%%% Поля и разметка страницы %%%
\usepackage{geometry}   % Для последующего задания полей

%%% Математические пакеты %%%
\usepackage{amsthm,amsmath,amscd}    % Математические дополнения от AMS
\usepackage{amsfonts,amssymb}        % Математические дополнения от AMS
%\usepackage{mathtools}              % Добавляет окружение multlined
%\usepackage{xfrac}                  % Красивые дроби

\usepackage{xcolor}


%%% Кодировки и шрифты %%%

\usepackage{cmap}                    % Улучшенный поиск русских слов в полученном pdf-файле
\usepackage[T1,T2A]{fontenc}         % Поддержка русских букв

%%% Гиперссылки %%%

\usepackage[unicode]{hyperref}[2012/11/06]
\usepackage[russian]{cleveref}


% Реализация пакетом biblatex через движок biber

%%% Реализация библиографии пакетами biblatex и biblatex-gost с использованием движка biber %%%
\usepackage[autostyle]{csquotes}

\usepackage[%
backend=biber,% движок
bibencoding=utf8,% кодировка bib файла
sorting=none,% настройка сортировки списка литературы
% defernumbers=true, % откомментируйте, если требуется правильная нумерация ссылок на литературу в режиме черновика. Замедляет сборку
]{biblatex}[2016/09/17]%
\addbibresource{external.bib}

% для списка литературы в диссертации и списка чужих работ в автореферате:
\newcommand{\bibtitlefull}{Список литературы} % (ГОСТ Р 7.0.11-2011, 4)

%%% Переопределение именований %%%
\renewcommand{\contentsname}{Оглавление}% (ГОСТ Р 7.0.11-2011, 4)
\renewcommand{\figurename}{Рисунок}% (ГОСТ Р 7.0.11-2011, 5.3.9)
\renewcommand{\tablename}{Таблица}% (ГОСТ Р 7.0.11-2011, 5.3.10)
\renewcommand{\listfigurename}{Список рисунков}%
\renewcommand{\listtablename}{Список таблиц}%
\renewcommand{\bibname}{\bibtitlefull}%