%%% Шаблон %%%
\DeclareRobustCommand{\fixme}{\textcolor{red}}  % решаем проблему превращения
%\newcommand{\fixme}{\textcolor{red}}  % решаем проблему превращения
% названия цвета в результате \MakeUppercase,
% http://tex.stackexchange.com/a/187930,
% \DeclareRobustCommand protects \fixme
% from expanding inside \MakeUppercase

%%% Макет страницы %%%
% Выставляем значения полей (ГОСТ 7.0.11-2011, 5.3.7)
\geometry{a4paper, top=2cm, bottom=2cm, left=2.5cm, right=1cm, nofoot, nomarginpar} %, heightrounded, showframe
\setlength{\topskip}{0pt}   %размер дополнительного верхнего поля
\setlength{\footskip}{12.3pt} % снимет warning, согласно https://tex.stackexchange.com/a/334346

%%% Интервалы %%%
%% По ГОСТ Р 7.0.11-2011, пункту 5.3.6 требуется полуторный интервал
%% Реализация средствами класса (на основе setspace) ближе к типографской классике.
%% И правит сразу и в таблицах (если со звёздочкой)
%\DoubleSpacing*     % Двойной интервал
\OnehalfSpacing*    % Полуторный интервал
%\setSpacing{1.42}   % Полуторный интервал, подобный Ворду (возможно, стоит включать вместе с предыдущей строкой)

%%% Выравнивание и переносы %%%
%% http://tex.stackexchange.com/questions/241343/what-is-the-meaning-of-fussy-sloppy-emergencystretch-tolerance-hbadness
%% http://www.latex-community.org/forum/viewtopic.php?p=70342#p70342
\tolerance 1414
\hbadness 1414
\emergencystretch 1.5em % В случае проблем регулировать в первую очередь
\hfuzz 0.3pt
\vfuzz \hfuzz
%\raggedbottom
%\sloppy                 % Избавляемся от переполнений
\clubpenalty=10000      % Запрещаем разрыв страницы после первой строки абзаца
\widowpenalty=10000     % Запрещаем разрыв страницы после последней строки абзаца
\brokenpenalty=4991     % Ограничение на разрыв страницы, если строка заканчивается переносом

%%% Оглавление %%%
\renewcommand{\cftchapterdotsep}{\cftdotsep}                % отбивка точками до номера страницы начала главы/раздела


\AtBeginDocument{%
	\setlength{\parindent}{2.5em}                   % Абзацный отступ. Должен быть одинаковым по всему тексту и равен пяти знакам (ГОСТ Р 7.0.11-2011, 5.3.7).
}

\usepackage{cmap}   % Улучшенный поиск русских слов в полученном pdf-файле
    \usepackage[T1,T2A]{fontenc}                    % Поддержка русских букв

  
\renewcommand{\chaptername}{Глава}
