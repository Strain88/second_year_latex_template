\chapter{Простая постановка задачи}\label{ch:ch1}

\section{Цилиндр с радиальным изменением температуры}\label{sec:ch1/sec1}

Рассмотрим сплошной цилиндр с осью симметрии \(oz\) и радиусом \(b\) и с законом изменения радиальной температуры выраженной \(\theta(r)=T(r)-T_0\), где \(T_0\)~--- начальная температура. Предположим также плоскюу деформацию \(\epsilon_{zz}=\epsilon_{rz}=\epsilon_{\phi z}=0\). Тогда напряженно-деформированное состояние запишется в виде (записываем сразу в цилиндрических координатах и продолжаем работать в них):
\begin{equation}
	\label{eq:ch1:equation1}
\begin{split}
	\epsilon_{rr} &= \frac{1}{E} \big [\sigma_{rr} - \nu \big(\sigma_{\phi\phi} + \sigma_{zz}\big) \big] + \alpha \theta \\
	\epsilon_{\phi\phi} &= \frac{1}{E} \big [\sigma_{\phi\phi} - \nu \big(\sigma_{rr} + \sigma_{zz}\big) \big] + \alpha \theta \\
	\sigma_{zz} &= \nu \big(\sigma_{rr} + \sigma_{\phi\phi}\big) - E\alpha\theta \\
\end{split}
\end{equation}	

Выразим напряжения через деформации, получим
\begin{equation}
	\label{eq:ch1:equation2}
	\begin{split}
		\sigma_{rr} = \frac{E}{(1+\nu)(1-2\nu)} \big[\big(1-\nu \big)\epsilon_{rr} + \nu\epsilon_{\phi\phi} - \big(1+\nu\big )\alpha\theta\big] \\
		\sigma_{\phi\phi} = \frac{E}{(1+\nu)(1-2\nu)} \big[\big(1-\nu \big)\epsilon_{\phi\phi} + \nu\epsilon_{rr} - \big(1+\nu\big )\alpha\theta\big] 
	\end{split}
\end{equation}

Уравнение равновесия для осесимметричной задачи выглядит следующим образом:
\begin{equation}
	\label{eq:ch1:equation3}
		\frac {d\sigma_{rr} }{dr} + \frac {\sigma_{rr}-\sigma_{\phi\phi}}{r}=0
\end{equation}
при этом деформации выраженные через радиальные перемещения \(u\) выглядят так:
\begin{equation}
	\label{eq:ch1:equation4}
	\epsilon_{rr} = \frac{du}{dr} \qquad \epsilon_{\phi\phi} = \frac{u}{r}
\end{equation}

Уравнения~\cref{eq:ch1:equation4} подставим в \cref{eq:ch1:equation2}, что дает следующее:
\begin{equation}
	\label{eq:ch1:equation5}
	\begin{split}
		\sigma_{rr} = \frac{E}{(1+\nu)(1-2\nu)} \big[\big(1-\nu \big)\frac{du}{dr} + \nu\frac{u}{r} - \big(1+\nu\big )\alpha\theta\big] \\
		\sigma_{\phi\phi} = \frac{E}{(1+\nu)(1-2\nu)} \big[\big(1-\nu \big)\frac{u}{r} + \nu\frac{du}{dr} - \big(1+\nu\big )\alpha\theta\big] 
	\end{split}
\end{equation}

Подставляя полученные уравнения \cref{eq:ch1:equation5} в уравнение равновесия \cref{eq:ch1:equation3} и немного упрощая, получим уравнение равновесия в радиальных перемещениях \(u\):
\begin{equation}
	\label{eq:ch1:equation6}
	\frac {d}{dr} \big[\frac{1}{r} \frac{d \big(ur \big)}{dr} \big] = \frac{1+\nu}{1-\nu}\alpha\frac{d\theta}{dr}
\end{equation}

Интегрирование \cref{eq:ch1:equation6} дает следующее уравнение:
\begin{equation}
	\label{eq:ch1:equation7}
	u = \frac{1+\nu}{1-\nu} \frac{\alpha}{r} \int_0^r \theta rdr +C_1r +\frac{C_2}{r}
\end{equation}
где \(C_1\) и \(C_2\) постоянные интегрирования. Поскольку перемещения должны быть конечными в центре (\(r=0\)), отсюда следует, что \(C_2\) должно быть равным нулю. Тогда компоненты перемещения в \cref{eq:ch1:equation4} принимаю вид:
\begin{equation}
	\label{eq:ch1:equation8}
	\begin{split}
		\epsilon_{rr} &= \frac{1+\nu}{1-\nu} \frac{\alpha}{r^2} \int_0^r \theta rdr +C_1 + \frac{1+\nu}{1-\nu} \alpha\theta\\
		\epsilon_{\phi\phi} &= \frac{1+\nu}{1-\nu} \frac{\alpha}{r^2} \int_0^r \theta rdr +C_1
	\end{split}
\end{equation}

и напряжения из уравнения \cref{eq:ch1:equation2} превращяются:

\begin{equation}
	\label{eq:ch1:equation9}
	\begin{split}
		\sigma_{rr} &= -\frac{E}{1-\nu} \frac{\alpha}{r^2} \int_0^r \theta rdr +C_1 \frac{E}{\big(1+\nu\big)\big(1-2\nu\big)}\\
		\sigma_{\phi\phi} &= -\frac{E}{1-\nu} \frac{\alpha}{r^2} \int_0^r \theta rdr -\frac{E \alpha \theta}{1-\nu} +C_1\frac{E}{\big(1+\nu\big)\big(1-2\nu\big)}
	\end{split}
\end{equation}

Определим константуу \(C_1\) используя граничные условия
\begin{equation}
	\label{eq:ch1:equation10}
	\sigma_{rr} = 0 \quad \text{на границе} \quad r=b
\end{equation}
что приводит к следующему
\begin{equation}
	\label{eq:ch1:equation11}
	C_1 = \frac{\alpha \big(1+\nu\big) \big(1-2\nu\big)}{\big(1-\nu\big) b^2} \int_0^b \theta rdr
\end{equation}

После подстановки \cref{eq:ch1:equation7} в \cref{eq:ch1:equation9}, получили:
\begin{equation}
	\label{eq:ch1:equation12}
	\begin{split}
		u &= \frac{1+\nu}{1-\nu} \frac{\alpha}{r} \big[ \int_0^r \theta rdr +\big(1-2\nu\big ) \frac{r^2}{b^2}\int_0^b \theta r dr\big]\\
		\sigma_{rr} &= \frac{E \alpha}{1-\nu}\big[ \frac{1}{b^2}\int_0^b \theta rdr -\frac{1}{r^2}\int_0^r \theta r dr\big]\\
		\sigma_{\phi\phi} &= \frac{E \alpha}{1-\nu}\big[ \frac{1}{b^2}\int_0^b \theta rdr +\frac{1}{r^2}\int_0^r \theta r dr - \theta \big]
	\end{split}
\end{equation}

Напряжения в осевом направлении, \(\sigma_{zz}\), получим из \cref{eq:ch1:equation1}

\begin{equation}
	\label{eq:ch1:equation13}
	\sigma_{zz} = \frac{E \alpha}{1-\nu}\big[ \frac{2\nu}{b^2}\int_0^b \theta rdr  - \theta \big]
\end{equation}	
	
Для тонкостенного цилиндра с радиусами \(a\) и \(b\), определяющие уравнения для перемещений записываются в этих границах --- от \(a\) внутреннего радиуса, до \(r\). Из \cref{eq:ch1:equation7}

\begin{equation}
	\label{eq:ch1:equation14}
	u = \frac{1+\nu}{1-\nu} \frac{\alpha}{r} \int_a^r \theta rdr +C_1r +\frac{C_2}{r}
\end{equation}

Подставляя \(u\) из \cref{eq:ch1:equation4} \cref{eq:ch1:equation2}, радиальные напряжения \(\sigma_{rr}\) находятся следующим образом:

\begin{equation}
	\label{eq:ch1:equation15}
	\sigma_{rr} = E \big [-\frac{\alpha}{\big(1-\nu \big) r^2} \int_a^r \theta rdr + \frac{C_1}{\big (1+\nu\big ) \big(1-2\nu \big)} - \frac{C_2}{\big(1+\nu\big) r^2} \big]
\end{equation}


Применим граничные условия
\begin{equation}
	\label{eq:ch1:equation16}
	\begin{split}
		\sigma_{rr} = 0 \quad \text{на границе} \quad r=a\\
		\sigma_{rr} = 0 \quad \text{на границе} \quad r=b
	\end{split}
\end{equation}

приводит к следующему

\begin{equation*}
	\begin{split}
		C_1 &= \frac{\big(1+\nu \big)\big(1-2\nu \big)}{\big(1-\nu \big)}  \frac{\alpha}{\big(b^2-a^2 \big)} \int_a^b \theta rdr\\
		C_2 &= \frac{\big(1+\nu \big)}{\big(1-\nu \big)}\frac{\alpha a^2}{\big(b^2-a^2 \big)}\int_a^b \theta rdr
	\end{split}
\end{equation*}

Подставим \(C_1\) и \(C_2\) в \cref{eq:ch1:equation14}, радиальные перемещения и напряжения получаются следующими:

\begin{equation}
	\label{eq:ch1:equation17}
	\begin{split}
		u &= \frac{1+\nu}{1-\nu} \frac{\alpha}{r} \big[ \frac{\big( 1-2\nu \big) r^2 +a^2}{b^2-a^2} \int_a^b \theta rdr + \int_a^r \theta r dr\big]\\
		\sigma_{rr} &= \frac{E \alpha}{1-\nu}\big[ \frac{1}{b^2-a^2} \big(1-\frac{a^2}{r^2} \big)\int_a^b \theta rdr -\frac{1}{r^2}\int_a^r \theta r dr\big]\\
		\sigma_{\phi\phi} &= \frac{E \alpha}{1-\nu}\big[ \frac{1}{b^2-a^2}\big(1+\frac{a^2}{r^2} \big)\int_a^b \theta rdr +\frac{1}{r^2}\int_a^r \theta r dr - \theta \big]
	\end{split}
\end{equation}	
	
Осевые напряжения из \cref{eq:ch1:equation1}	получаются следующими:

\begin{equation}
	\label{eq:ch1:equation18}
	\sigma_{zz} = \frac{E \alpha}{1-\nu}\big[ \frac{2\nu}{b^2-a^2}\int_a^b \theta rdr  - \theta \big]	
\end{equation}

и осевая сила \(F_z\) в этом случае, \(\epsilon_{zz}=0\)

\begin{equation}
	\label{eq:ch1:equation19}
	F_z = \int_a^b 2 \pi r \sigma_{zz} dr
\end{equation}


Если положить внутреннюю температуру в цилиндре \(T_a\), а внешнюю \(T_b\), то распределение температуры можно записать в виде:

\begin{equation}
	\label{eq:ch1:equation20}
	{\color{red}T=\frac{T_a - T_b}{\ln{\frac{b}{a}}} \big(\ln{\frac{b}{r}} \big) + T_b}
\end{equation}

\[T_a - T_b = T_d\]
{\color{red}TODO: объяснить откуда это пришло?}

Подставляя температурное распределение в напряжения для полого цилиндра с фиксированными гранями \cref{eq:ch1:equation17}, получаем:

\begin{equation}
	\label{eq:ch1:equation21}
	\begin{split}
		\sigma_{rr} &= \frac{E \alpha T_d}{2 \big (1 -\nu \big) \ln{\frac{b}{a}}}\big[ \ln{\frac{b}{r}} + \frac{a^2}{b^2-a^2} \big(1-\frac{b^2}{r^2} \big) \ln{\frac{b}{a}}\big]\\
		\sigma_{\phi\phi} &= \frac{E \alpha T_d}{2 \big (1-\nu \big ) \ln{\frac{b}{a}} }\big[1 - \ln{\frac{b}{r}} - \frac{a^2}{b^2-a^2}\big(1+\frac{b^2}{r^2} \big)\ln{\frac{b}{a}} \big] \\
		\sigma_{zz} &= \frac{\nu E \alpha T_d}{2 \big (1-\nu \big) \ln{\frac{b}{a}}} \big [1-\frac{2 a^2}{b^2 - a^2}\ln{\frac{b}{a}}-\frac{2}{\nu} \ln{\frac{b}{r}} \big] - E \alpha \big(T_b - T_0 \big)
	\end{split}
\end{equation}

Теперь рассмотрим случай обобщенного условия деформации в плоскости для полого толстого цилиндра. Из уравнения ({\color{red}Сослаться на 1.19}) следует, что осевая нагрузка равна нулю

\begin{equation}
	\label{eq:ch1:equation22}
	F_z = \int_a^b 2 \pi r \sigma_{rr} dr = 0
\end{equation}

Если переписать осевую деформацию цилинда \(\epsilon_{zz}\) с нагревом в полярных координатах, то получим

\begin{equation}
	\label{eq:ch1:equation23}
	\sigma_{zz} = r \big (\sigma_{rr}+\sigma_{\phi\phi} \big) + E \alpha (\bar{\theta} - \theta)
\end{equation}

где \(\theta = T - T_0\) и

\begin{equation}
	\label{eq:ch1:equation24}
	\bar{\theta} = \frac{1}{A} \int_A dA = \frac{2 \pi}{\pi \big (b^2 - a^2 \big)} \int_a^b \theta rdr = \frac{2}{b^2-a^2} \int_a^b \theta rdr
\end{equation}


Подставляя компоненты напряжения из \cref{eq:ch1:equation17} в \cref{eq:ch1:equation23} получаем:

\begin{equation}
	\label{eq:ch1:equation25}
	\sigma_{zz} = \frac{E \alpha}{1 - \nu} \big ( \bar{\theta} - \theta \big )
\end{equation}

По-другому, сложив компоненты напряжений (радиальные и тангенсальные) из \cref{eq:ch1:equation17}:

\begin{equation}
	\label{eq:ch1:equation26}
	\sigma_{rr} + \sigma_{\phi\phi} = \frac{E \alpha}{1-\nu} \big [\frac{2}{b^2 - a^2} \int_a^b \theta rdr - \theta \big] = \frac{E \alpha}{1 - \nu} \big ( \bar{\theta} - \theta \big )
\end{equation}

Для этого особого случая
\begin{equation}
	\label{eq:ch1:equation27}
	\sigma_{rr} + \sigma_{\phi\phi} = \sigma_{zz}
\end{equation}

Когда толстостенный цилиндр также подвергается внутреннему и внешнему давлению, \(p_a\) и \(p_b\), соответственно, полное напряжение в цилиндре становится суммой тепловых и механических напряжений. Механические напряжения получаются из тех же управляющих уравнений, за исключением того, что член \(T\) исчезает. Таким образом, последовательное решение этих уравнений для радиального перемещения \(u\) дается уравнением \cref{eq:ch1:equation7} за исключением члена, включающего интеграл от
температуры, который исчезает, а радиальное перемещение становится равным:

\begin{equation}
	\label{eq:ch1:equation28}
	u = C_1 r + \frac{C_2}{r}
\end{equation}

Подставляя это перемещение в \cref{eq:ch1:equation4}, а потом в \cref{eq:ch1:equation2} при следующих граничных условиях

\begin{equation}
	\label{eq:ch1:equation29}
	\begin{split}
		\sigma_{rr} = -p_a \quad \text{на границе} \quad r=a \\
		\sigma_{rr} = -p_b \quad \text{на границе} \quad r=b
	\end{split}
\end{equation}

даёт уравнения радиальных и тангенсальных напряжений для механического нагружения:

\begin{equation}
	\label{eq:ch1:equation30}
	\begin{split}
		\sigma_{rr} = \frac{p_a a^2}{b^2 - a^2} \big (1-\frac{b^2}{r^2} \big )- \frac{p_b b^2}{b^2-a^2} \big (1 - \frac{a^2}{r^2} \big )\\
		\sigma_{\phi\phi} = \frac{p_a a^2}{b^2 - a^2} \big (1+\frac{b^2}{r^2} \big )- \frac{p_b b^2}{b^2-a^2} \big (1 + \frac{a^2}{r^2} \big )
	\end{split}
\end{equation}

Как вывод, когда толстый цилиндр подвергается как тепловым, так и механическим напряжения, результирующие напряжения являются суммой уравнений \cref{eq:ch1:equation21} и \cref{eq:ch1:equation30}.

\section{Неравномерный нагрев цилиндра}
\label{sec:ch1/sec2}
Надеюсь до этого не дойдет. Тут беда. 
Если дойдет, то посмотреть на метод косплексных переменных 

\chapter{Функционально-градиентные материалы}\label{ch:ch2}

Функционально-градиентные материалы (ФГМ) --- это новые передовые жаропрочные материалы, используемые в современных технологиях. Помимо превосходных тепловых свойств и способности выдерживать сверхвысокие термические нагрузки, они обладают коррозионной и
устойчивы к эрозии и обладают высокой прочностью на излом.


Исторически метод создания ФГМ состоит в том, чтобы смешать керамику и металл таким образом, чтобы свойства материала непрерывно изменялись от одного составляющего материала к другому. Фактически, коэффициенты определяющих уравнений для распределения температуры и напряжения зависят от координат, поскольку свойства материала являются функциями положения.

Будем рассматривать степенное изменения свойств материала.

\section{Толстый цилиндр из функционально-градиентного материала} \label{ch:ch2/sec1}

Рассмотрим толстый полый цилиндр внутреннего радиуса \(a\) и внешнего радиуса \(b\) изготовленный из ФГМ. Материал цилиндра имеет градацию по направлению \(r\), поэтому свойства материала являются функциями \(r\). Пусть \(u\) и \(v\) --- компоненты перемещения в радиальном и окружном направлениях, соответственно. Тогда соотношения между деформацией и перемещением имеют вид:

\begin{equation}
	\label{eq:ch2:equation1}
	\begin{split}
		\epsilon_{rr} & = u_{,r}\\
		\epsilon_{\phi\phi} &= \frac{u_{,\phi}}{r} + \frac{u}{r}\\
		\epsilon_{r \phi} &= \frac{1}{2} \big(\frac{u_{,\phi}}{r} - \frac{v}{r} \big)
	\end{split}
\end{equation}
где значок \((,)\) обозначает частную производную. Тогда напряженно-деформированное состояние для плоской деформации запишется в виде:

\begin{equation}
	\label{eq:ch2:equation2}
	\begin{split}
		\sigma_{rr} &= \big(\lambda + 2 \mu \big) \epsilon_{rr} + \lambda \epsilon_{\phi\phi} - \big(3 \lambda +2 \mu \big) \alpha \theta (r, \phi)\\
		\sigma_{\phi\phi} &= \big(\lambda + 2 \mu \big) \epsilon_{\phi \phi} + \lambda \epsilon_{rr} - \big(3 \lambda +2 \mu \big) \alpha \theta (r, \phi)\\
		\sigma_{r \phi} &= 2\mu \epsilon_{r \phi}
	\end{split}
\end{equation}

где \(\sigma_{ij}\) и \(\epsilon_{ij}\) (\(i,j = r,\phi)\) --- тензоры напряжений и деформаций, \(\theta (r, \phi) \) --- распределение температуры, определяемое уравнением теплопроводности, где \(\alpha \) --- коэффициент линейного теплового расширения, а \( \lambda \) и \(\mu \) --- коэффициенты Ламэ.

Уравнения равновесия в радиальном и окружном направлениях, без сосредоточенных сил и сил инерции, имеют вид


\begin{equation}
	\label{eq:ch2:equation3}
	\begin{split}
		\sigma_{rr,r} &+ \frac{1}{r} \sigma_{r \phi, \phi} + \frac{1}{r} \big ( \sigma_{rr} - \sigma_{\phi\phi}\big ) = 0 \\
		\sigma_{r \phi, r} &+ \frac{1}{r} \sigma_{\phi \phi, \phi} + \frac{2}{r} \sigma_{r \phi} = 0
	\end{split}
\end{equation}


Для получения уравнений равновесия в перемещениях
для ФГМ-цилиндра, закон распределения внутренних свойств материала должен быть известен заранее. Поскольку предполагается, что материал цилиндра меняется в \(r\) --- радиальном направлении --- модуль упругости и коэффициент теплового расширения предполагается описывать степенными законами как

\begin{equation}
	\label{eq:ch2:equation4}
	\begin{split}
		E(r) &= E_0 \big ( \frac{r}{l} \big) ^ {m_1} \\
		\alpha (r) &= \alpha_0 \big ( \frac{r}{l} \big) ^{ m_2}
	\end{split}
\end{equation}

где \(E_0\) и \(\alpha_0 \) --- постоянные материала, а \(m_1\) и \(m_2\) --- степенные показатели распределения свойств, \(l\) --- характерная длина. Здесь и далее положим,
что коэффициент Пуассона постоянен.

Подставляя \cref{eq:ch2:equation1} в \cref{eq:ch2:equation4}, уравнения Навье в перемещения записываются в следующем виде.
\begin{adjustwidth}{3.5em}{3.5em} В общем случае напряжения можно выразить через деформации, а затем и перемещения. Если, после взятия частной производной, подставить это в уравнения равновесия, то получатся следующие уравнения, которые называется уравнениеми Навье. В ПДСК это выглядит следующим образом:

\begin{equation*}
\begin{split}
	&\sigma_{ij,j} + X_i = \rho \ddot{u_i} \quad \text{уравнение равновесия} \\
	&\sigma_{ij} = \mu \big (u_{i,j} + u_{j,i} \big ) + \big [ \lambda u_{k,k} - \alpha \big (3 \lambda +2\mu \big ) \big (T-T_0 \big ) \big ] \delta_{ij}\\
	&\mu u_{i,kk} + \big ( \lambda + \mu \big ) u_{k,ki} - \big ( 3 \lambda + 2\mu \big ) \alpha T_{,i} +X_i = \rho \ddot{u_i} \quad \text{уравнение Навье}
\end{split}
\end{equation*}
\end{adjustwidth}


{\color{red} ВОТ ТУТ НУЖНО БЫ ЕЩЕ РАЗИК ПРОВЕИТЬ :

\begin{equation}
	\label{eq:ch2:equation5}
	\begin{split}
		u_{,rr} &+ \big ( m_1+1\big ) \frac{1}{r} u_{,r} + \big ( \frac{\nu m_1}{1-\nu} -1 \big ) \frac{1}{r^2} u + \big ( \frac{1-2\nu}{2-2\nu} \big) \frac{1}{r^2} u_{\phi \phi} + \big ( \frac{1}{2-2\nu}\big ) \frac{1}{r} v_{,r \phi} \\
		&+ \big [ \frac{\big ( 4+2 m_1\big ) \nu -3 }{2-2\nu} \big ] \frac{1}{r^2} v_{, \phi} = \frac{\big ( 1+\nu\big ) \alpha_0 }{\big (1-\nu \big ) l^{m_2}} \Big [ \big (m_1 + m_2 \big ) r^{m_2 -1} \theta + r^{m_2} \theta_{,r}\Big] \\
		%
		v_{,rr} &+ \big (m_1 +1 \big ) \frac{1}{r} v_{,r} - \big (m_1 + 1 \big ) \frac{1}{r^2} v + \big ( \frac{2-2\nu }{1-2\nu } \big ) \frac{1}{r^2} v_{,\phi \phi} + \big ( \frac{1}{1-2\nu} \big ) \frac{1}{r} u_{,r \phi} \\
		&+ \big ( \frac{3-4\nu}{1-2\nu} + m_1\big ) \frac{1}{r^2} u_{, \phi} = \big (\frac{2+2\nu}{1-2\nu} \big ) \big (\frac{\alpha_0 r^{m_2 -1}}{l^{m_2}} \big ) \theta_{,\phi}
	\end{split}
\end{equation}
}


Чтобы получить компоненты перемещения \(u\) и \(v\), распределение температуры должно быть известно. Используя груничные условия ({\color{red} описать ГУ}), закон распределения температруры запишется в виде:

\begin{equation}
	\label{eq:ch2:equation6}
	\theta(r, \phi) = \sum_{n=-\infty}^{+\infty} \big (A_{n1} r^{\beta_{n1}} + A_{n2} r^{\beta_{n2}} \big ) e^{in \phi}
\end{equation}

где \(\beta_{n1}\) и \(\beta_{n2}\) и константы интегрирования \(A_{n1}\) и \(A_{n2}\) определены ранее {\color{red} вот раньше из ГУ} 

TEST