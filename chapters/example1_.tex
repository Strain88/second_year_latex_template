\chapter{Простой пример нагруженного <<толстого>> цилиндра} \label{ch:example1}

Рассмотрим вот такой пример...
Допустим, есть толстостенные цилиндр с внотренними и внешними радиусами \(a=1\,m\) и \(b=1.2\,m\) и высотой \(l~=~1\,m\). Коэффициент Пуассона \(\mu=0.3\) и модуль упругости и коэффициент термального расширения на внутреннем радиусе \(E_i = 200\,Gpa\) и \(\alpha_i = 1.2 \times 10^{-6} /^{\circ}C\) задаются уравнением \cref{eq:ch2:equation4} и коэффициент теплопроводности записывается в таком же виде \( k(r) = k_0 \big ( \frac{r}{l} \big ) ^ {m_3} \). Для простоты степенная функция распределения свойст материалов будет одинаковым \(m_1=m_2=m_3=m\).

На внутренней границе отсутсвует сила, только задана температура по следующему закону:
\begin{equation}
\label{eq:example1:1}
	T(a, \phi) = 60 cos (3\phi^{\circ}C)
\end{equation}
 --- не осесимметричное нагружение. Внешняя граница закреплена --- запрещены перемещения в радиальном направлении с нулевой температурой. 
\begin{equation}
\label{eq:example1:2}
\begin{split}
	\sigma_{rr}(a, \phi) &= 0,\\
	\sigma_{r\phi}(a, \phi) &= 0,\\
	u(b, \phi) &= 0,\\
	v(b, \phi) &= 0
\end{split}
\end{equation}

Чтобы получить температурное распределение, как \cref{eq:ch2:equation6}, определеним константы интегрирования \(A_1, A_2\) подставив температуру в дифференциальное уравнение теплопроводности (в цилиндрической СК).

\begin{equation}
\label{eq:example1:3}
	k \left(\frac{\partial^2 T}{\partial r^2} +\frac{1}{r}\frac{\partial T}{\partial r} +\frac{1}{r^2}\frac{\partial^2 T}{\partial \phi^2} +\frac{\partial^2 T}{\partial z^2} \right) +R = \rho c \frac{\partial T}{\partial t}
\end{equation}

Что в случае отсутсвия источников тепла, нагружения (правой части) и оси симметрии выразится в следующем уравнении. Рассматриваем полностью установившееся равновесие:

\begin{equation}
\label{eq:example1:4}
	\frac{\partial^2 T}{\partial r^2} + \frac{1}{r} \frac{\partial T}{\partial r} +\frac{1}{r^2} \frac{\partial ^2 T}{\partial \phi^2} = 0
\end{equation}