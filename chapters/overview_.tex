\chapter{Обзор существующей проблемы}\label{ch:overview:1}

\section{Обзор современного состояния и ключевых проблем проектирования, потенциальных решений для конструкций резервуаров для хранения криогенного жидкого водорода для применения в авиации}\label{ch:overview:1:sec1}
% https://ntrs.nasa.gov/api/citations/20060056194/downloads/20060056194.pdf

\subsection{Абстракт}\label{ch:overview:1:sec1:sub1}

Благодаря высокой удельной энергоемкости жидкий водород (\(LH_2\) --- liquid hydrogen~\(H_2\)) выделяется на горизонте альтернативных источников энергии для будущих летательных аппаратов. Как следствие, существует потребность в системах хранения водорода, которые должны обеспечивать достаточную вместимость для полетов продолжительностью от нескольких минут до нескольких дней. Понятно, что разработка большого, легкого, многоразового криогенного резервуара для хранения водорода имеет решающее значение для достижения целей и обеспечения топливом летательных аппаратов на водороде, особенно для длительных полетов. В данной работе  представлен обзор текущего состояния дел в области материалов, конструкций и систем изоляции криогенных резервуаров ---- наряду с попыткой определения ключевых проблем  разработки легкой и долговременной системы хранения \(LH_2\). Рассмотренные широкие классы изоляционных систем включают пенопласты (включая современные аэрогели) и системы многослойной изоляции (МСИ -- MLI --- multilayer insulation) с вакуумом. Системы МСИ показывают перспективность для долгосрочного применения. Рассмотренные структурные конфигурации включают одно- и двустенные конструкции, в том числе многослойные конструкции. Потенциальными кандидатами для материала стенок являются монолитные металлы, а также полимерные матричные композиты и прерывисто армированные металлические матричные композиты. Для применения в кратковременных полетах может быть достаточно простых конструкций баков. В качестве альтернативы для более длительных полетов наиболее оптимальной представляется конструкция с двойными стенками и вакуумной системой изоляции. Современные тенденции в разработке материалов для обшивки рассматриваются в случае, когда обшивка требуется для минимизации или устранения потерь водородного топлива через проницаемость.


Интерес к разработке летательных аппаратов, использующих альтернативные источники энергии, такие как водород, обусловлен прежде всего тем, что водород обеспечивает низкий или нулевой выброс в окружающую среду вредных продуктов. Среди рассматриваемых вариантов применения -- продолжительность полета, которая может составлять от нескольких минут до многих дней. Как пример, вот пример современного беспилотного летательного аппарата (БПЛА)  NASA с большой продолжительностью полета: Helios~HP03, БПЛА на солнечных батареях, использующий систему регенеративных топливных элементов для накопления энергии. Он был способен летать в течение месяца, но имел ограниченную грузоподъемность 230 кг (550 фунтов), которая должна была распределяться по крыльям, и мог летать на пиковой высоте около 21 км (70 000 футов).

В коммерческих самолетах продолжительность полета, скорее всего, будет составлять порядка нескольких часов. Стремление к увеличению грузоподъемности и продолжительности полета требует использования силовой установки с более высокой удельной мощностью и повышенной общей эффективностью. Исследуемые в настоящее время системы включают использование топливных элементов с электродвигателями и двигателями внутреннего сгорания. Текущие предварительные требования к программам, которые стимулируют разработку водородных самолетов с большой продолжительностью полета, включают продолжительность полета 14 дней (336 часов) с полезной нагрузкой, достаточной для размещения приборов.

Водород обладает наибольшей энергией на единицу массы среди различных видов жидкого и газообразного топлива, как отмечает Томас (\cite{thomas}). Водород, хранящийся в жидком виде, значительно увеличивает энергию на единицу объема по сравнению с газообразным водородом (\(GH_2\) --- gaseous hydrogen). Газообразный водород, хранящийся при давлении 35~МПа~(5~ksi) и температуре \(20^{\circ} C \,(68^{\circ} F)\), хранит только одну треть энергетического содержания на единицу объема жидкого водорода (\(LH_2\)), как показано Томасом. Хотя для хранения (\(LH_2\)) при низком давлении и криогенной температуре требуется изоляция, она меньше, чем при хранении  (\(GH_2\)) при высоком давлении. Другой метод хранения водорода в компактной и безопасной форме --- это гидрид металла. К сожалению, использование гидридов металлов накладывает ограничение, связанное с чрезмерным весом, что исключает их использование в чувствительных к весу приложениях. 

В настоящее время применение криогенных резервуаров для хранения в аэрокосмической отрасли, где вес имеет первостепенное значение, ограничено короткими полетами, например, на космических ракетах-носителях. 
Криогенные жидкости переливаются в баки для хранения транспортного средства непосредственно перед запуском, а большая часть жидкостей расходуется во время выхода на орбиту, в течение нескольких минут. 
Криогенные жидкости исчерпываются со скоростью, при которой выкипание не представляет значительной проблемы (\( {\color{red} \text{уточнить у кого-нибудь, кто в теме}} \)). В таких случаях для резервуаров обычно достаточно легкой пенопластовой изоляции. 
В глубоком космосе теплопередача к криогенной жидкости значительно меньше, чем в условиях окружающей среды на поверхности Земли, что снижает необходимость в сверхнизкой проводимости и толстой изоляции. Именно для самолетов с относительно большой продолжительностью полета порядка нескольких дней возникают наибольшие инженерные трудности при разработке долговременных и легких систем хранения водорода.

Необходимость снижения веса в сочетании с хорошими изоляционными свойствами для долгосрочного хранения представляет собой новую задачу в проектировании криогенных резервуаров. Это дает возможность применить более современные материалы и конструктивные идеи в попытке снизить общий вес и сохранить объем на приемлемом и практичном уровне.

В данном работе рассматриваются конструктивные и тепловые элементы системы баков для хранения криогенных веществ для летательного аппарата. В следующих разделах будут рассмотрены отдельные компоненты системы баков. 

После подробного описания основных проблем в следующих разделах будут рассмотрены подкомпоненты системы баков. Также будут рассмотрены материалы и их термическая и химическая совместимость со средами, в которых работает система хранения \(LH_2\). Будут рассмотрены методы строительства резервуара. Сюда входит оценка металлических и полимерно-матричных композитных (ПМК) материалов и архитектуры, используемой для создания резервуара. Также будет обсуждаться возможность использования футеровки ({\color{green}  Футеровка (нем. Futter «подкладка, подбой») -- облицовка огнеупорными, химически стойкими, износостойкими, а также теплоизоляционными материалами, которым покрывается внутренняя поверхность металлургических печей, ковшей, топок котлов и прочего оборудования. Футеровка производится для обеспечения защиты поверхностей от возможных механических, термических, физических и химических повреждений.}).

Другие важные области конструкции бака, включая методы крепления для интеграции системы бака с планерной рамой, ребра жесткости, перегородки для выброса топлива и порты, подробно не рассматриваются, поскольку они выходят за рамки данного отчета.

Методы изоляции будут рассмотрены для определения оптимальной системы для применения в летательных аппаратах. 

\section{Ключевые проблемы разработки}\label{ch:overview:1:sec2}

Успешная реализация будущих легко-весных и экологически чистых ЛА требует объединения ряда инновационных и передовых технологий, а также разработки новых технических средств и технологий их производства. Как уже упоминалось выше, особый интерес представляют ЛА на водородном топливе с продолжительностью полета в несколько дней. Водород можно хранить как в газообразной, так и в жидкой форме. Газообразный водород требует в \(5.6\) раз большего объема по сравнению с жидкой форме, при этом жидкий водород хранится под давлением примерно \(163\) атм (\(2400\) фунтов на кв. дюйм) и при температуре \(15^{\circ}C (60^{\circ}F)\).

Необходимый при этом чрезмерный объём, вес бака и проблемы связанные с обеспечением безопасности, при с хранением водорода под высоким давлением, исключают использование газообразного водорода \(GH_2\) в качестве топлива для данного применения. В качестве альтернативы, гидриды металлов могут компактно и безопасно хранить водород, хотя эти системы будут слишком тяжелыми и, следовательно, непрактичными. Кроме того, для <<перезарядки>> большинства металлогидридных систем потребуются дни или даже недели. Гидриды \(MgNi\) могут достичь приблизительно 5 процентов водорода по весу, но требуют более высоких температур для активации кристаллического слоя.

Проблемы металлогидридов были обобщены Брюером (\cite{brewer1991}), который отметил их непрактичность для авиационных применений. Жидкий водород в насыщенном состоянии является жизнеспособным вариантом для самолетов с большой продолжительностью полета. Однако проектирование криогенного резервуара для хранения \(LH_2\), в сочетании с использованием \(LH_2\) в качестве авиационного топлива, сопряжено со многими проблемами. Некоторые из ключевых проблем, включая геометрию, температуру, проницаемость, водородное охрупчивание и факторы безопасности, кратко обсуждаются ниже.

\subsection{Реальные размеры, геометрия}\label{ch:overview:1:sec2:sub1}

Несмотря на то, что жидкий водород \(LH_2\) имеет высокую температуру сгорания, его низкая плотность в сочетании с потенциально большой продолжительностью полета и использованием нерекуперативной системы хранения может привести к необходимости бака большого объема.

Неинтегральный (т.е. не являющийся частью конструкции летательного аппарата) большой бак означает большую лобовую (поперечную) площадь и площадь поверхности, что приведет к увеличению сопротивления. Для успешных конфигураций потребуются инновационные конструкции с малой лобовой площадью и площадью поверхности.

Интеграция большого бака в ЛА сама по себе является серьезной проблемой. Интегральные баки, напротив, требуют сложной конструкции и создают множество производственных проблем. Интегральные баки являются конструктивным элементом ЛА, т.е. воспринимают нагрузки на фюзеляж, а также обеспечивают сохранность топлива. Неинтегральные баки служат только в качестве контейнеров для топлива и устанавливаются внутри корпуса ЛА и удерживаются им (\cite{brewer1991}). Как упоминалось ранее, неинтегральные баки не обязательно должны соответствовать форме ЛА; их конструкция может быть относительно простой, например, сферической или цилиндрической. Баки сферической формы обеспечивают минимальную площадь поверхности для данного объема, поэтому пассивное нагревание бака может быть сведено к минимуму, что минимизирует выкипание LH2. Однако сферическая форма создает некоторые специфические производственные трудности и имеет большую площадь фронтальной поверхности, что приводит к увеличению силы сопротивления по сравнению с баком цилиндрической формы. В качестве альтернативы, цилиндрические формы проще в изготовлении, но имеют более высокое отношение площади поверхности к объему, что приводит к более высокой пассивной тепловой нагрузке на бак.

\subsection{Криогенные температуры}\label{ch:overview:1:sec2:sub2}

Нормальная температура кипения \(LH_2\) составляет \(-252^{\circ}C (-423^{\circ}F)\). \(LH_2\) необходимо держать ниже этой температуры, чтобы минимизировать выкипание, то есть потерю топлива и повышение давления в баке. Во время наземных операций максимальная разница температур внутри и снаружи конструкции бака может достигать \(\Delta T = 300^{\circ}C~(540^{\circ}F)\). Для поддержания такого большого температурного градиента, очевидно, необходима легковесная изоляция с низкой теплопроводимостью. Как уже упоминалось, большинство предыдущих применений криогенного топлива, такого как \(LH_2\), было ограничено короткой продолжительностью в несколько минут, как в ракетах-носителях, где проблема выкипания топлива не является критической. Однако для полетов большой продолжительности чрезмерное выкипание топлива очень нежелательно и ограничивает продолжительность полета летательного аппарата. Следовательно, количество пассивного тепла, которое поступает в бак и вызывает выкипание жидкого водорода, должно быть ограничено. Хранение \(LH_2\) на земле имеет аналогичные проблемы и требует специализированного оборудования и процедур для работы с ним. Установлено, что если поддерживать постоянное абсолютное давление \(LH_2\) на уровне примерно \(170 кПа (25 psia)\), то выкипание будет поддерживаться на приемлемом уровне и без лишних весовых нагрузок на конструкцию резервуара (\cite{brewer1991}). Кроме того, резервуар и любые соединительные трубопроводы или приспособления должны быть полностью изолированы от внешней атмосферы, поскольку все газы, за исключением гелия, застывают при температурах LH2 и повышают вероятность образования закупорки проточных трубопроводов и других компонентов.

\subsection{Проницаемость}\label{ch:overview:1:sec2:sub3}

Поскольку молекулы водорода очень малы, они чрезвычайно легко проникают через стенку резервуара. Проницаемость водорода является, пожалуй, самым критическим моментом в конструкции резервуара. Металлические баки являются очевидным решением этой проблемы, поскольку водород проникает через металлы медленнее, чем сквозь неметаллические материалы. Однако для летательного аппарата масса металлического бака может ограничить его грузоподъемность и продолжительность полета. Бак из композита с полимерной матрицей (КПМ --- PMC polymer matrix composite) с тонкой металлической обшивкой также решил бы проблему проницаемости, но вес все еще может оказаться проблемой. Кроме того, несоответствие коэффициента теплового расширения (КТР) между стенкой композитного бака и металлическим вкладышем приведет к их неравномерному сжатию, что, в результате, может вызвать напряжения в материале, которые могут привести к отслойке вкладыша от бака и/или его разрушению, что делает такую конструкцию нежелательной. 

Исследования проницаемости водорода, проведенные в ходе программы "Национальный аэрокосмический самолет" ( National Aerospace Plane program), были обнадеживающими, и было показано, что композитные баки без какого-либо вкладыша достаточно непроницаемы. Однако разрушение бака КПМ \(LH_2\) демонстрационного проекта X-33 во время наземных испытаний было вызвано микротрещинами полимерной матрицы в композитной внутренней оболочке конструкции бака (\cite{grimsley2001}). Микротрещины в композите возникли в результате несоответствия КТР углеродного волокна и полимерной матрицы в сочетании с большой разницей между температурой использования и температурой изготовления композита. Микротрещины создавали путь для утечки или проникновения водорода под давлением через стенку и проникновения в сотовый заполнитель. При нагревании трещины в матрице закрывались, жидкость испарялась, а образовавшиеся газы, которым некуда было выходить, вызывали повышение давления и в конечном итоге отслоение сердечника от внутренней композитной оболочки.  В последнее время были проведены исследования по оценке полимерных пленок и покрытий, которые можно было бы нанести на внутреннюю композитную оболочку и использовать в качестве барьера для удержания \(LH_2\) в резервуаре. Изготовление легких и непроницаемых резервуаров из современных материалов для криогенного применения является сложной задачей.

\subsection{Водородное охрупчивание}\label{ch:overview:1:sec2:sub4}

Многие материалы при воздействии водорода в больших концентрациях становятся хрупкими. Влияние водорода на поведение материалов хорошо описано (например, \cite{moodythompson1990}). Это один из видов разрушения материала. Когда материал становится более хрупким, его несущая способность и пластичность снижаются. Таким образом, могут произойти катастрофические разрушения без значительной деформации или видимого деградирования детали конструкции. Это ограничивает применение многих современных материалов в строительстве стенки резервуара.

Для решения этих проблем необходимы прорывные технологии в области материалов. Для возникновения водородного охрупчивания необходимы растягивающие напряжения, восприимчивый к воздействию водорода материал и присутствие водорода. Водородное охрупчивание может привести к образованию трещин при уровнях напряжения значительно ниже предела текучести. Хотя водородное охрупчивание больше всего зафиксировано для высокопрочных сталей, все материалы обладают определенной степенью восприимчивости. Устойчивость к водородному охрупчиванию должна быть важным фактором при выборе материалов для стенок резервуара.

\subsection{Коэффициент безопасности}\label{ch:overview:1:sec2:sub5}

Использование требуемых коэффициентов безопасности в диапазоне от \(1.4\) до \(2.0\), которые обычно дополняются консервативными расчетами на прочность материала (такими как использования допустимых значений на основе А-базиса), очень затрудняет достижение облегченной конструкции. Это особенно актуально, когда в строительстве таких конструкций используются нетрадиционные передовые материалы. Кроме того, новые и/или современные материалы не очень хорошо изучены, особенно при экстремально низких температурах, а процессы производства и изготовления вносят дополнительную вариативность в свойства материала. Все это приводит к значительному разбросу свойств материалов, что требует значительной разницы между средними измеренными и допустимыми значениями.

Будущие конструкции резервуаров, безусловно, потребуют инновационных разработок, выверенных с помощью испытаний и включающих интегрированные методы мониторинга состояния здоровья для снижения явных и неявных коэффициентов безопасности.  В целом, поиск правильного баланса между (1) минимизацией веса прочной конструкции бака, вмещающей необходимое количество топлива, (2) выживанием в течение необходимого количества циклов полета, включающих циклы заправки и слива топлива и соответствующие термомеханические нагрузки, и (3) созданием конструкции, которая может быть действительно изготовлена, проверена и использована с уверенностью.


\section{Предыдущий опыт использования креогенных баков}\label{ch:overview:1:sec3}

Исследования в области хранения \(LH_2\) для летательных аппаратов и космических кораблей проводились в течение многих лет. Поскольку летательные аппараты и космические аппараты имеют схожие требования, ниже приводится краткий обзор прошлых разработок криогенных баков для авиакосмической техники в хронологическом порядке. Также приводится краткое описание других применений, включая мобильные и стационарные наземные хранилища. 

\subsection{Самолеты с водородными двигателями на заре авиации}\label{ch:overview:1:sec3:sub1}

Одно из самых ранних задокументированных примеров использования криогенного резервуара для хранения водорода и заправки водородным топливом летательного аппарата было представлено Холом и Сильверштейном (\cite{hallsilverstein1955}) и Рейнольдсом (\cite{reynolds1955}). Двухмоторный бомбардировщик ВВС США B-57 был модифицирован таким образом, чтобы один из двигателей работал на водороде. Это было частью программы Исследовательского центра НАСА имени Гленна (NASA Glenn Research Center), Кливленд, штат Огайо (бывший Исследовательский центр Льюиса), для демонстрации характеристик сгорания водорода в авиационных газотурбинных установках, как отмечает Брюер (\cite{brewer1991}). Следует отметить, что основной упор в программе был сделан на успешную работу газотурбинного двигателя на водороде; следовательно, не было уделено внимание разработке легкой системы криогенных резервуаров.


\subsection{Сатурн 5. <<Saturn V>>}\label{ch:overview:1:sec3:sub2}

Ракета Saturn V --- американская сверхтяжёлая ракета-носитель семейства Сатурн, использовавшеяся в 1960-х и 1970-х годах. В ракете Saturn V использовались алюминиевые баки с пенопластовой изоляцией. Одностенный металлический бак с изоляцией из пенопласта достаточно показателен для типичных ракетоносителей, использующих жидкий водород в качестве топлива для запуска с Земли на орбиту. То, что ракета-носитель была запущена с Земли на орбиту, обуславливает такое хранение как краткосрочное.

Две верхние ступени двигателя Saturn V использовали \(LH_2\) вместе с жидким кислородом (\(LO_2\)) в качестве топлива. Глейзер (\cite{glaser1967}) показал вторую ступень, состоящую из алюминиевой стенки, окруженной сотовым заполнителем из пенопласта, затем нейлоновым фенольным покрытием и последним слоем пластиковой пленки Tedlar (DuPont) для защиты от аэродинамического нагрева. Пенопласт продувался гелием. Глейзер описал третий этап как использование пенопласта на внутренней стороне стенки резервуара. Расположение пенопласта уменьшило необходимость в специальных методах и клеях для крепления пенопласта. В результате стенка бака работала при более высокой температуре и не нуждалась в защите от аэродинамического нагрева. На внутренней поверхности изоляции был установлен паро- и жидкостный барьеры.

\subsection{Внешний бак космического челнока. <<Space Shuttle>>}\label{ch:overview:1:sec3:sub3}

С начала 1980-х годов и по настоящее время НАСА использует космический челнок в качестве тяжелого ракето-носителя. Внешний бак космического челнока (ВБ) представляет собой современную легкую конструкцию криогенного хранилища, которая используется в настоящее время. Внешний бак является несущей конструкцией в системе космического челнока. Он предназначен для хранения \(LH_2\) и \(LO_2\), нагрузки от самого орбитального корабля и нагрузки от твердотопливных ракетных ускорителей (боковой ускоритель, SRB ---  solid rocket boosters). Кроме того,  криогенные жидкости хранятся под давлением около 2 атм. Внешний бак --- это расходуемый компонент, который заполняется непосредственно перед запуском и используется в течение короткого промежутка времени при выведении шаттла со стартовой площадки на низкую околоземную орбиту примерно за \(8.5\)~мин. Вес хранимого \(LH_2\) составляет \(1.0\) МН (228 000 фунтов) со скоростью выкипания примерно \(4.4\) Н/с (1 фунт/с) при стабилизации бака (П. Роджерс, 2006, NASA Marshall Space Flight Center, Huntsville, AL, личное сообщение), что соответствует примерно \(1.6\,\%\) от веса \(LH_2\) в час.  Часть внешнего бака, находящаяся под давлением, изготовлена из алюминиево-литиевого сплава 2195. Первоначально использовался алюминиевый сплав 2219, который позже был заменен на алюминиево-литиевый сплав 2195, как описано Бикли и Швингхамером (\cite{bickleyschwinghame1999}). Алюминиево-литиевый сплав обеспечивает увеличение прочности и небольшое снижение плотности по сравнению с ранее использовавшимся сплавом 2219.  ВБ представляет собой алюминиевую полумонококовую конструкцию из сваренных плавлением стволовых секций, пяти основных кольцевых рам, а также носового и кормового эллипсоидных куполов, согласно Petty (\cite{petty2006}). Тепловая защита обеспечивается напыляемой пеной и предварительно формованными материалами абляторов. На веб-сайте НАСА, посвященном ВБ, Петти дает хороший обзор бака.


\subsection{X-30 National Aero-Space Plane. <<NASP>>}\label{ch:overview:1:sec3:sub4}

Совместные усилия Министерства обороны США (МО) и НАСА по разработке Национального аэрокосмического самолета (NASP) в течение большей части 1980-х годов включали попытку разработать легкие криогенные баки для одноступенчатого аппарата для полета с Земли на орбиту. Некоторые из усилий NASP были обобщены в работах Cope и Thorndyke (\cite{copethorndyke1992}) и Hellwig et al. (\cite{hellwig1992}). Изначально NASP был исследованием целесообразности создания одноступенчатого транспортного средства, способного взлетать и приземляться горизонтально, как кратко описано в работе Jenkins et al. (\cite{jenkins2003}). Потенциальный прототип получил обозначение X-30. Позднее рамки программы были изменены для разработки гиперзвукового межконтинентального самолета. 

Жесткие требования, предъявляемые к одноступенчатым и гиперзвуковым межконтинентальным аэрокосмическим аппаратам, повышали важность снижения веса. В проекте NASP использовался многослойный композитный бак для хранения жидкого водорода. Это была ранняя попытка использования КПМ с криогенными жидкостями для снижения веса по сравнению с более распространенными металлическими конструкциями. Бак был изготовлен из тонкостенного композита с эпоксидной матрицей, армированного углеродным волокном, с элементами жесткости, как описано в Hellwig et al. (\cite{hellwig1992}). Кроме того, в баке использовались внутренние ограничители для уменьшения деформации стенок бака при нагнетании давления (\cite{lohmuelle2006}) и отсутствовала обшивка (\cite{robinson1994}).

\subsection{Первый летательный аппарат, работающий только на водороде}\label{ch:overview:1:sec3:sub5}

Брюер (\cite{brewer1991}) описал первый самолет, полностью работающий на водородном топливе. Полет состоялся 19 июня 1988 года. Предыдущие самолеты использовали несколько двигателей, и только один двигатель работал на водороде. В самолете, описанном Брюером, для хранения LH2 использовался цилиндрический бак с эллипсоидными торцевыми крышками. Бак был изготовлен из нержавеющей стали типа 304. Емкость была установлена внутри алюминиевого внешнего корпуса и опиралась в нескольких точках на распорки с низкой теплопроводностью. Пространство между баком и оболочкой было вакуумировано для минимизации теплопроводности, а для минимизации радиационной теплопередачи внутри вакуумированного пространства использовалась многослойная изоляция. Однако общая цель программы заключалась в демонстрации полета с использованием исключительно двигателя на водородном топливе, а не в разработке легкого бака для хранения LH2 в течение длительного времени. 

\subsection{Lockheed Martin X-33}\label{ch:overview:1:sec3:sub6}

Проект X-33 был естественным продолжением проекта NASP (Х-30). Программа NASA X-33 была важным проектом, в котором использовались передовые материалы и концепции для усовершенствования технологии криогенных баков-аккумуляторов. Суборбитальный корабль Х-33 был разработан для демонстрации передовых технологий, которые должны были значительно повысить надежность ракеты-носителя и снизить стоимость выведения полезной нагрузки на низкую околоземную орбиту. X-33 должен был стать средством демонстрации технологий для VentureStar. Большая часть работы над X-33 была обобщена в отчете Центра космических полетов НАСА имени Джорджа К. Маршалла (\cite{nasamarshall2000}).

На аппарате X-33 было два топливных бака LH2, каждый размером 8,7 на 6,1 на 4,3 м (28,5 на 20 на 14 футов). Исследования водородной проницаемости, проведенные в ходе программы NASP, были обнадеживающими, и считалось, что композитные баки без какого-либо вкладыша достаточно непроницаемы. На Х-33 использовалась более сложная трехслойная сендвич-конструкция баков. Многослойная конструкция состояла из графит-эпоксидных внутренней и внешней обшивок и графит-эпоксидного невентилируемого сердечника Hexcel Composites, как описано в Dornheim (\cite{dornheim1999}). Баки состояли из трех основных субкомпонентов: кормового купола и переборки, стволовой секции и носового купола и переборки. Баки также являлись неотъемлемой структурной частью основного корпуса корабля и должны были находиться в очень сложном напряженном состоянии. Эти баки были испытаны в Центре космических полетов имени Маршалла НАСА в Хантсвилле, штат Алабама, в ноябре 1999 года для проверки структурной целостности баков LH2 при криогенной температуре и различных условиях давления и механических нагрузок, предполагаемых как типичные для использования в аппарате Х-33.

После успешного завершения первого испытания прототипа на давление и нагрузку, из испытуемого изделия был слит LH2 и началась продувка бака. Примерно через 15 минут после слива жидкости из бака произошла авария: внешняя лицевая панель и сердцевина одной из долей отделились от внутренней лицевой панели. Было установлено, что все процедуры испытания и параметры находились в пределах конструктивных ограничений испытуемого изделия.

Была собрана следственная группа, которая назвала наиболее вероятной причиной отказа сочетание следующих явлений (отчет NASA Marshall, \cite{nasamarshall2000}):  (1) Микротрещины во внутреннем слое лицевой панели с последующей просачиванием жидкого водорода (2) Попадание внешнего азотистого газа в вакуум и последующее разжижение при контакте с криогенной границей (3) Снижение предела прочности и жесткости клеевой линии (4) Производственные браки и дефекты (5) Просачивание жидкого водорода в наполнитель панели, что привело к более высокому, чем ожидалось, давлению в сотовой панели.

Последний фактор стал неожиданной причиной возникновения разрушения. Конструкция бака расширила границы возможностей  и объединила множество непроверенных технологических элементов, создав очень сложную систему. Использование замкнутого композитного бака не только для перевозки топлива, но и для передачи механических нагрузок было довольно смелым решением для того времени. Процесс производства выявил некоторые сложности, связанные с масштабированием больших композитных конструкций, которые не были изучены ранее. Главным уроком, извлеченным из этого опыта, стало явление и значение микротрещин в композитных панелях криогенных баков под воздействием тепловых и механических нагрузок. Отказ испытательного аппарата X-33 во время наземных испытаний был обобщен в работе Grimsley et al. (\cite{grimsley2001}), которая показала, что причиной отказа стало проникновение водорода в сердцевину композитной многослойной структуры через микротрещины матрицы.


Тем не менее, группа по расследованию аварий (отчет NASA Marshall, \cite{nasamarshall2000}) отметила, что результаты исследования не опровергают использование композитных материалов для криогенных баков. 


Уроки, извлеченные из этого испытания, если их применить к технологии композитных криогенных баков с точки зрения проектирования и производства, должны способствовать развитию технологии и успешному использованию композитов для будущих криогенных баков. 

 Ближе к концу программы было принято решение продолжить использование водородного бака из алюминиевого сплава для испытательного аппарата X-33 после того, как возникли трудности с баком КПМ. Программа X-33 выявила те области, которые требуют дальнейшего изучения и развития, включая решение проблемы микротрещинообразования композитов, использование вкладышей и других деталей конструкции стенок бака.

\subsection{Резервуар для жидкого водорода в программе Next Generation Launch Technology}\label{ch:overview:1:sec3:sub7}

Программа Next Generation Launch Technology (NGLT) была продолжением программы X-33 по разработке следующего поколения многоразовых аэрокосмических аппаратов. Программа NGLT основывалась на опыте X\nobreakdash-33. Предварительный анализ концепции бака был выполнен Абумери и др. (\cite{abumeri2004}) в рамках программы NASA NGLT. Концепция конструкции состояла из бака КПМ с тонкими стенками и дополнительными внутренними слоями ламината, образующими продольные ребра жесткости. Другое исследование Робинсона и др. (\cite{robinson2004}) иллюстрировало исследования по определению конструкции бака для следующей многоразовой ракеты-носителя. Результаты этих исследований привели к созданию композитного бака из углеродного волокна с сотовым сердечником и многослойной структурой, используемой на X-33 и в проекте NGLT. После проведения дополнительных исследований в рамках проекта NGLT была введена защитная пленка на внутренней стенке бака.

В продолжение проекта криогенного бака X-33 NASA и Northrop Grumman разработали и завершили испытания уменьшенного композитного криогенного бака в Marshall. Эти баки были ключевым частью всех многоразовых элементов программы NASA NGLT. Эти баки также были составной частью планера; то есть они несли стартовые нагрузки и нагрузки на крыло в дополнение к топливу. Четвертьмасштабный цилиндрический бак длиной 4,6 м (15 футов) и диаметром 1,8 м (6 футов) был опрессован под давлением около 779 кПа (113 фунтов на кв. дюйм).

Это тестовое давление примерно в четыре раза превышает эксплутационное давление в реальном полноразмерном баке, чтобы вызвать аналогичные напряжения стенок, возникающие при запуске из-за нагрузок от жидкости, ускорения и т.д. Согласно планам испытаний, бак должен был быть охлажден, наполнен и подвергнут нагрузке около 40 раз в течение нескольких месяцев (Glass, \cite{glass2004}, и Sharke, \cite{sharke2004}). Согласно пресс-релизу подрядчика NASA - компании Northrop Grumman - эти испытания были успешно завершены (McKinney and Neiwert, \cite{mckinneyneiwert2004}).

\subsection{Другие сферы применения криогенных хранилищ}\label{ch:overview:1:sec3:sub8}

В настоящее время ведутся исследования по созданию недорогих легких криогенных хранилищ для наземного транспорта. Наземные транспортные средства имеют требования к долгосрочному хранению, аналогичные тем, которые необходимы для гражданских самолетов. Министерство энергетики США (DOE) уже давно работает с водородом в качестве топлива для транспортных систем. Одна из попыток включала создание автомобиля, работающего на LH2, как представил Стюарт (\cite{stewart1982}). В этой работе использовался бак-накопитель более традиционной конструкции, то есть металлический бак с изоляцией MLI. Масса системы является важным вопросом для наземного транспорта, такого как автомобили, но не в такой степени, как для воздушных и космических транспортных средств. Общая стоимость, как правило, является значительной движущей силой наряду с требованиями к малому весу.

Другие автомобильные проекты кратко описаны в DaimlerChrysler (\cite{daimlerchrysle2004}). Пример автомобильного криогенного бака представлен Magna Steyr (\cite{magnasteyr2006}), который является частью совместных усилий BMW AG, Linde AG и Magna Steyr и состоит из двустенного бака, изготовленного из аустенитной стали. В объеме между стенками используется MLI и поддерживается высокий вакуум 10-9 бар. Это похоже на ранний проект DOE, представленный Стюартом (\cite{stewart1982}).

Стационарные наземные хранилища не имеют строгих ограничений по весу и размеру, налагаемых подвижными системами. Таким образом, такие резервуары также могут быть изготовлены из аустенитной стали, окруженной вакуумной защитной оболочкой с MLI. Кроме того, для долгосрочного хранения можно использовать холодильную установку. Однако системы баков для хранения водорода, используемые в самолетах, требуют использования чрезвычайно легких конструкций. Необходимость создания легких, хорошо изолированных систем хранения, пригодных для полетов, представляет собой настоящую инженерную задачу. 

Следует отметить, что многие исследования и разработки, включая NASP, X-33 и NGLT, проводились для приложений, требующих относительно короткой продолжительности работы. Короткая продолжительность позволяет увеличить потери из-за утечек и выкипания по сравнению с длительной продолжительностью полета водородного самолета. Многие из вышеупомянутых вариантов применения сфокусированы на силовых установках, работающих на водороде. В результате в опубликованных отчетах были представлены минимальные подробности структурных конструкций систем хранения, используемых в этих приложениях. Хотя вышеупомянутые программы ракет-носителей продвинули вперед технологию легких криогенных баков-аккумуляторов, все еще существует необходимость в разработке системы изоляции бака, которая была бы легкой и способной удерживать утечку и выкипание на приемлемом уровне. 
