\chapter{Обзор существующей проблемы}\label{ch:overview}
% https://ntrs.nasa.gov/api/citations/20060056194/downloads/20060056194.pdf

\section{Обзор современного состояния и ключевых проблем проектирования, потенциальных решений для конструкций резервуаров для хранения криогенного жидкого водорода для применения в авиации}\label{sec:overview/sec1}

Благодаря высокой удельной энергоемкости жидкий водород (\(LH_2\) --- liquid hydrogen~\(H_2\)) выделяется на горизонте альтернативных источников энергии для будущих летательных аппаратов. Как следствие, существует потребность в системах хранения водорода, которые должны обеспечивать достаточную вместимость для полетов продолжительностью от нескольких минут до нескольких дней. Понятно, что разработка большого, легкого, многоразового криогенного резервуара для хранения водорода имеет решающее значение для достижения целей и обеспечения топливом летательных аппаратов на водороде, особенно для длительных полетов. В данной работе  представлен обзор текущего состояния дел в области материалов, конструкций и систем изоляции криогенных резервуаров ---- наряду с попыткой определения ключевых проблем  разработки легкой и долговременной системы хранения \(LH_2\). Рассмотренные широкие классы изоляционных систем включают пенопласты (включая современные аэрогели) и системы многослойной изоляции (МСИ -- MLI --- multilayer insulation) с вакуумом. Системы МСИ показывают перспективность для долгосрочного применения. Рассмотренные структурные конфигурации включают одно- и двустенные конструкции, в том числе многослойные конструкции. Потенциальными кандидатами для материала стенок являются монолитные металлы, а также полимерные матричные композиты и прерывисто армированные металлические матричные композиты. Для применения в кратковременных полетах может быть достаточно простых конструкций баков. В качестве альтернативы для более длительных полетов наиболее оптимальной представляется конструкция с двойными стенками и вакуумной системой изоляции. Современные тенденции в разработке материалов для обшивки рассматриваются в случае, когда обшивка требуется для минимизации или устранения потерь водородного топлива через проницаемость.


Интерес к разработке летательных аппаратов, использующих альтернативные источники энергии, такие как водород, обусловлен прежде всего тем, что водород обеспечивает низкий или нулевой выброс в окружающую среду вредных продуктов. Среди рассматриваемых вариантов применения -- продолжительность полета, которая может составлять от нескольких минут до многих дней. Как пример, вот пример современного беспилотного летательного аппарата (БПЛА)  NASA с большой продолжительностью полета: Helios~HP03, БПЛА на солнечных батареях, использующий систему регенеративных топливных элементов для накопления энергии. Он был способен летать в течение месяца, но имел ограниченную грузоподъемность 230 кг (550 фунтов), которая должна была распределяться по крыльям, и мог летать на пиковой высоте около 21 км (70 000 футов).

В коммерческих самолетах продолжительность полета, скорее всего, будет составлять порядка нескольких часов. Стремление к увеличению грузоподъемности и продолжительности полета требует использования силовой установки с более высокой удельной мощностью и повышенной общей эффективностью. Исследуемые в настоящее время системы включают использование топливных элементов с электродвигателями и двигателями внутреннего сгорания. Текущие предварительные требования к программам, которые стимулируют разработку водородных самолетов с большой продолжительностью полета, включают продолжительность полета 14 дней (336 часов) с полезной нагрузкой, достаточной для размещения приборов.

Водород обладает наибольшей энергией на единицу массы среди различных видов жидкого и газообразного топлива, как отмечает Томас (\cite{thomas}). Водород, хранящийся в жидком виде, значительно увеличивает энергию на единицу объема по сравнению с газообразным водородом (\(GH_2\) --- gaseous hydrogen). Газообразный водород, хранящийся при давлении 35~МПа (5~ksi) и температуре \(20^{\circ} C \,(68^{\circ} F)\), хранит только одну треть энергетического содержания на единицу объема жидкого водорода (\(LH_2\)), как показано Томасом. Хотя для хранения (\(LH_2\)) при низком давлении и криогенной температуре требуется изоляция, она меньше, чем при хранении  (\(GH_2\)) при высоком давлении. Другой метод хранения водорода в компактной и безопасной форме --- это гидрид металла. К сожалению, использование гидридов металлов накладывает ограничение, связанное с чрезмерным весом, что исключает их использование в чувствительных к весу приложениях. 

В настоящее время применение криогенных резервуаров для хранения в аэрокосмической отрасли, где вес имеет первостепенное значение, ограничено короткими полетами, например, на космических ракетах-носителях. 
Криогенные жидкости переливаются в баки для хранения транспортного средства непосредственно перед запуском, а большая часть жидкостей расходуется во время выхода на орбиту, в течение нескольких минут. 
Криогенные жидкости исчерпываются со скоростью, при которой выкипание не представляет значительной проблемы (\( {\color{red} \text{уточнить у кого-нибудь, кто в теме}} \)). В таких случаях для резервуаров обычно достаточно легкой пенопластовой изоляции. 
В глубоком космосе теплопередача к криогенной жидкости значительно меньше, чем в условиях окружающей среды на поверхности Земли, что снижает необходимость в сверхнизкой проводимости и толстой изоляции. Именно для самолетов с относительно большой продолжительностью полета порядка нескольких дней возникают наибольшие инженерные трудности при разработке долговременных и легких систем хранения водорода.

Необходимость снижения веса в сочетании с хорошими изоляционными свойствами для долгосрочного хранения представляет собой новую задачу в проектировании криогенных резервуаров. Это дает возможность применить более современные материалы и конструктивные идеи в попытке снизить общий вес и сохранить объем на приемлемом и практичном уровне.

В данном работе рассматриваются конструктивные и тепловые элементы системы баков для хранения криогенных веществ для летательного аппарата. В следующих разделах будут рассмотрены отдельные компоненты системы баков. 

После подробного описания основных проблем в следующих разделах будут рассмотрены подкомпоненты системы баков. Также будут рассмотрены материалы и их термическая и химическая совместимость со средами, в которых работает система хранения \(LH_2\). Будут рассмотрены методы строительства резервуара. Сюда входит оценка металлических и полимерно-матричных композитных (ПМК) материалов и архитектуры, используемой для создания резервуара. Также будет обсуждаться возможность использования футеровки ({\color{green}  Футеровка (нем. Futter «подкладка, подбой») -- облицовка огнеупорными, химически стойкими, износостойкими, а также теплоизоляционными материалами, которым покрывается внутренняя поверхность металлургических печей, ковшей, топок котлов и прочего оборудования. Футеровка производится для обеспечения защиты поверхностей от возможных механических, термических, физических и химических повреждений.}).

Другие важные области конструкции бака, включая методы крепления для интеграции системы бака с планерной рамой, ребра жесткости, перегородки для выброса топлива и порты, подробно не рассматриваются, поскольку они выходят за рамки данного отчета.

Методы изоляции будут рассмотрены для определения оптимальной системы для применения в летательных аппаратах. 