\chapter{Обзор существующей проблемы}\label{ch:overview}
% https://ntrs.nasa.gov/api/citations/20060056194/downloads/20060056194.pdf

\section{Обзор современного состояния и ключевых проблем проектирования, потенциальных решений для конструкций резервуаров для хранения криогенного жидкого водорода для применения в авиации}\label{sec:overview/sec1}

Благодаря высокой удельной энергоемкости жидкий водород (LH2) становится альтернативным топливом для будущих летательных аппаратов. . Как следствие, существует потребность в системах хранения водородных баков для этих летательных аппаратов, которые должны обеспечивать достаточную вместимость для полетов продолжительностью от нескольких минут до нескольких дней. Понятно, что разработка большого, легкого, многоразового криогенного резервуара для хранения водорода имеет решающее значение для удовлетворения потребностей в водородных резервуарах. резервуара для хранения жидкости имеет решающее значение для достижения целей и обеспечения энергией самолетов на водородном топливе, особенно для длительных полетов. В данном отчете представлен обзор с комментариями (включая результаты обширного обзор литературы) текущего состояния дел в области материалов, структурных конструкций и систем изоляции криогенных резервуаров. изоляционных систем - наряду с определением ключевых проблем - с целью разработки легкой и долговременной системы хранения LH2. Рассмотренные широкие классы изоляционных систем включают пенопласты (включая современные аэрогели) и системы многослойной изоляции (MLI) с вакуумом. На сайте Системы MLI показывают перспективность для долгосрочного применения. Рассмотренные структурные конфигурации включают одно- и двустенные конструкции, в том числе многослойные конструкции. Потенциальными кандидатами на материал для стенок являются монолитные металлы, а также полимерные матричные композиты и прерывисто армированные металлические матричные композиты. композиты. Для применения в кратковременных полетах может быть достаточно простых конструкций баков. В качестве альтернативы, для более длительных полетов, двустенная конструкция с вакуумной системой изоляции
представляется наиболее оптимальной конструкцией. Современные тенденции в разработке материалов для лайнеров рассматриваются в следующих случаях в том случае, если лайнер необходим для минимизации или устранения потерь водородного топлива через проницаемость.

\cite{einstein}